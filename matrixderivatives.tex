\documentclass[12pt,a4paper]{article}
\usepackage{amsmath}
\usepackage{amsthm}
\usepackage{amsfonts}
\usepackage{amssymb}
\usepackage{amsmath,amscd}
\usepackage[symbol]{footmisc}
\usepackage{fancyhdr}
\usepackage{graphicx}
\usepackage{bm}
\usepackage[english]{babel}
\linespread{1.25}


\DeclareMathOperator{\Tr}{Tr}
\DeclareMathOperator{\D}{D}
\DeclareMathOperator{\Real}{Re}
\DeclareMathOperator{\T}{\text{\scriptsize T}}

\renewcommand{\thefootnote}{\fnsymbol{footnote}}


\begin{document}

\section{Hamiltonian Monte Carlo for fuzzy spaces}
\subsection{Statement of the problem}
The fuzzy space action considered here is:
\begin{equation}
S[D] = g_2 \Tr D^2 + \Tr D^4
\end{equation} 
where $g_2 \in \mathbb{R}$ and $D$ is of the form:
\begin{equation}
D = \sum_i \omega_i \otimes (M_i \otimes I + \epsilon_i I \otimes M_i^T)
\end{equation}
for Hermitian $\omega_i$ and $M_i$, and $\epsilon_i = \pm 1$. \newline
The dynamical variables in the Monte Carlo are the $n \times n$ matrices $M_i$. Hamiltonian Monte Carlo requires to take derivatives such as:
\begin{equation}
\frac{\partial S[M_i]}{\partial M_k}
\end{equation}
which amounts to finding formulas for terms like:
\begin{equation}
\frac{\partial \Tr D^p}{\partial M_k}.
\end{equation}
In the following, formulas for $p=2$ and $p=4$ are developed.


\subsection{Matrix calculus}
Let $A \in M_n(\mathbb{C})$ and $f(A)$ be a complex valued function of $A$. The derivative of $f$ with respect to $A$ is defined in components as the $n \times n$ matrix:
\begin{equation}
\left(\frac{\partial f}{\partial A}\right)_{lm} \equiv \frac{\partial f}{\partial A_{lm}}.
\end{equation}
The two special cases of interest here are:
\begin{equation}
\frac{\partial \Tr A}{\partial A} = I
\end{equation}
\begin{equation}
\frac{\partial \Tr AB}{\partial A} = B^T.
\end{equation}


\subsection{The case $p=2$}
When $p = 2$ the $M_i$ matrices are decoupled:
\begin{equation}
\Tr D^2 = \sum_i \Tr \omega_i^2 (2 n \Tr M_i^2 + 2 \epsilon_i (\Tr M_i)^2).
\end{equation} 
Taking a derivative with respect to $M_k$ yields:
\begin{align}
\frac{\partial}{\partial M_k} \left( \sum_i \Tr \omega_i^2 (2 n \Tr M_i^2 + 2 \epsilon_i (\Tr M_i)^2) \right) = \notag \\
\sum_i \delta_{ik} \Tr \omega_i^2 \left( 4 n M_i^T + 4 \epsilon_i (\Tr M_i) I \right) = \notag \\
4 C \left( n M_k^T + \epsilon_k (\Tr M_k) I \right)
\end{align}
where $C \equiv \Tr \omega_i^2$ is the dimension of the Clifford module.

\subsection{The case $p=4$}
First expand $\Tr D^4$:
\begin{align}\label{eq:trd4}
\Tr D^4 &= \sum_{i_1, i_2, i_3, i_4} \Tr (\omega_{i_1} \omega_{i_2} \omega_{i_3} \omega_{i_4}) \cdot \notag \\
\Big( \ &n [1+\epsilon \ *] \Tr (M_{i_1} M_{i_2} M_{i_3} M_{i_4})  \ \  + \notag \\
&\epsilon_{i_1} \Tr M_{i_1} [1 + \epsilon \ *] \Tr ( M_{i_2} M_{i_3} M_{i_4})  \ \  + \notag \\
&\epsilon_{i_2} \Tr M_{i_2} [1  + \epsilon \ *] \Tr (M_{i_1} M_{i_3} M_{i_4}) \ \  + \notag \\
&\epsilon_{i_3} \Tr M_{i_3} [1 + \epsilon \ *] \Tr (M_{i_1} M_{i_2} M_{i_4})  \ \  + \notag \\
&\epsilon_{i_4} \Tr M_{i_4} [1 + \epsilon \ *] \Tr (M_{i_1} M_{i_2} M_{i_3})  \ \  + \notag \\
&\epsilon_{i_1} \epsilon_{i_2} [1 + \epsilon] \Tr (M_{i_1} M_{i_2} ) \Tr (M_{i_3} M_{i_4})   \ \  + \notag \\
&\epsilon_{i_1} \epsilon_{i_3} [1 + \epsilon] \Tr (M_{i_1} M_{i_3}) \Tr (M_{i_2} M_{i_4})   \ \  + \notag \\
&\epsilon_{i_1} \epsilon_{i_4} [1 + \epsilon] \Tr (M_{i_1} M_{i_4}) \Tr (M_{i_2} M_{i_3})   \ \Big)
\end{align}
where $*$ denotes complex conjugation of everything that appears on the right, $\epsilon$ is defined as the product $\epsilon \equiv \epsilon_{i_1}\epsilon_{i_2}\epsilon_{i_3}\epsilon_{i_4}$, and the relation $M^T = M^*$ has been used. The presence of $\epsilon = \pm 1$ inside the square brackets spoils the otherwise manifest reality of the expression, which instead comes from the fact that a simultaneous index exchange $i_1 \leftrightarrow i_4$ and $i_2 \leftrightarrow i_3$ is equivalent to taking the complex conjugate. \newline
Taking a matrix derivative with respect to $M_k$ results in non-vanishing contributions when $k=i_1, k=i_2, k=i_3$ or $k=i_4$:
\begin{align} \label{eq:derd4}
\frac{\partial}{\partial M_k} \Tr D^4 &= \sum_{i_1, i_2, i_3, i_4} \Tr (\omega_{i_1} \omega_{i_2} \omega_{i_3} \omega_{i_4}) \cdot \notag \\
\Big( &\delta_{ki_1} A(i_1, i_2, i_3, i_4)^T + \delta_{ki_2} A(i_2, i_3, i_4, i_1)^T + \notag \\
&\delta_{ki_3} A(i_3, i_4, i_1, i_2)^T + \delta_{ki_4} A(i_4, i_1, i_2, i_3)^T \Big)
\end{align}
where $A(a, b, c, d )$ is the following $n \times n$ matrix:
\begin{align}
A(a,b,c,d) \equiv \ &n [1+\epsilon \ \dagger] M_b M_c M_d + \notag \\
&\epsilon_a I [1 + \epsilon \ *] \Tr M_b M_c M_d  + \notag \\
&\epsilon_b \Tr M_b [1 + \epsilon \ \dagger] M_c M_d + \notag \\
&\epsilon_c \Tr M_c [1 + \epsilon \ \dagger] M_b M_d + \notag \\
&\epsilon_d \Tr M_d [1 + \epsilon \ \dagger] M_b M_c + \notag \\
&\epsilon_a \epsilon_b M_b [1 + \epsilon] \Tr M_c M_d + \notag \\
&\epsilon_a \epsilon_c M_c [1 + \epsilon] \Tr M_b M_d + \notag \\
&\epsilon_a \epsilon_d M_d [1 + \epsilon] \Tr M_b M_c
\end{align}
and $\dagger$ denotes Hermitian conjugation of everything that appears on the right. Again Eq.(\ref{eq:derd4}) does not look manifestly Hermitian because of the $\epsilon$ factor in the square brackets. To see that this is indeed the case, it is useful to write Eq.(\ref{eq:derd4}) symbolically as:
\begin{equation}
\frac{\partial}{\partial M_k} \Tr D^4 = \sum_{i_1, i_2, i_3, i_4} \mathcal{B}(i_1, i_2, i_3, i_4)
\end{equation}
and notice that a simultaneous exchange of indices $i_1 \leftrightarrow i_4$ and $i_2 \leftrightarrow i_3$ (call this permutation $\rho$) is equivalent to taking the Hermitian conjugate:
\begin{equation}\label{eq:BBd}
\mathcal{B}(i_1, i_2, i_3, i_4)^{\dagger} = \mathcal{B}(i_{\rho(1)}, i_{\rho(2)}, i_{\rho(3)}, i_{\rho(4)}) = \mathcal{B}(i_4, i_3, i_2, i_1).
\end{equation}
This property can also be exploited to gain some computational efficiency. One can introduce an equivalence relation
\begin{equation}
(i_1, i_2, i_3, i_4) \sim (i_{\rho^m(1)}, i_{\rho^m(2)}, i_{\rho^m(3)}, i_{\rho^m(4)}), \quad m \in \mathbb{N}
\end{equation}
and restrict the sum on the equivalence classes:
\begin{equation}\label{eq:trunc}
\frac{\partial}{\partial M_k} \Tr D^4 = \sum_{[i_1, i_2, i_3, i_4]} \frac{|[i_1, i_2, i_3, i_4]|}{2} [1+\dagger]\mathcal{B}(i_1, i_2, i_3, i_4)
\end{equation}
where $|[i_1, i_2, i_3, i_4]|$ is the cardinality of the class. Practically, $\rho^{2m}$ is the identity permutation, and the equivalence classes have one or two elements. The factor involving the cardinality of the class prevents from overcounting terms of the type $(a, b, b, a)$ (which are the only elements with class of cardinality one).\newline
The sum of Eq.(\ref{eq:trunc}) can be written explicitly in the following way:
\begin{align}\label{eq:almost}
\frac{\partial}{\partial M_k} \Tr D^4 &= \notag \\
&\sum_{\substack{i_4 < i_1 \\ i_3 < i_2}}[1+\dagger]\big(\mathcal{B}(i_1, i_2, i_3, i_4) + \mathcal{B}(i_1, i_3, i_2, i_4)\big) + \notag \\
&\sum_{\substack{i_4 = i_1 \\ i_3 < i_2}}[1+\dagger]\mathcal{B}(i_1, i_2, i_3, i_4) + \sum_{\substack{i_4 < i_1 \\ i_3 = i_2}}[1+\dagger]\mathcal{B}(i_1, i_2, i_3, i_4) + \notag \\
&\sum_{\substack{i_4 = i_1 \\ i_3 = i_2}}\mathcal{B}(i_1, i_2, i_3, i_4)
\end{align}
and simplified noticing that the second line vanishes due to the properties of the $\omega$ matrices:
\begin{equation}
\Tr(\omega_{\sigma(i_1)} \omega_{\sigma(i_2)} \omega_{\sigma(j)} \omega_{\sigma(k)}) \propto \Tr(\omega_j \omega_k) = 0 \qquad \text{if} \ \ i_1 = i_2 \ \text{and} \ j \neq k
\end{equation}
for any permutation $\sigma$ acting on $\{i_1, i_2, j, k\}$.\newline
For the same reason, terms in the first sum such that $(i_1 = i_2$ and $i_3 \neq i_4)$ or $(i_1 \neq i_2$ and $i_3 = i_4)$ need not be computed as they vanish.
Therefore Eq.(\ref{eq:almost}) simplifies to:
\begin{align}\label{eq:notquite}
\frac{\partial}{\partial M_k} \Tr D^4 =
&\sum_{i}\mathcal{B}(i, i, i, i) + \notag \\
&\sum_{i_1 \neq i_2}\big( \mathcal{B}(i_1, i_1, i_2, i_2) + \mathcal{B}(i_1, i_2, i_2, i_1) \big) + \notag \\
&\sum_{\substack{i_4 < i_1 \\ i_3 < i_2 \\ i_1 \neq i_2 \\ i_3 \neq i_4}}[1+\dagger]\big(\mathcal{B}(i_1, i_2, i_3, i_4) + \mathcal{B}(i_1, i_3, i_2, i_4)\big).
\end{align}

\subsection{A general formula for every $p$}
The first problem is to write $\Tr D^p$ in a useful form, along the lines of Eq.(\ref{eq:trd4}).\newline
$\Tr D^p$ expands to:
\begin{align}
\Tr D^p &= \sum_{i_1 \ldots i_p} \Tr \omega_{i_1} \ldots \omega_{i_p} \cdot \notag \\
&\Tr \big( (M_{i_1} \otimes I + \epsilon_{i_1} I \otimes M_{i_1}^T) \ldots  (M_{i_p} \otimes I + \epsilon_{i_p} I \otimes M_{i_p}^T) \big)
\end{align}
Ignoring (for now) the trace over the $\omega$ matrices, a typical term in the sum is:
\begin{equation}\label{eq:typical}
\Tr \big( \epsilon_B A \otimes B^* + \epsilon_A B \otimes A^* \big)
\end{equation}
where $A$ and $B$ are related to the product $M_{i_1} \ldots M_{i_p}$ in the following way:
\begin{enumerate}
\item pick $r \geq 0$ numbers $k_1 < \ldots < k_r$ from $\{1, \ldots , p\}$ and call the remaining $p-r$ numbers $j_1 < \ldots < j_{p-r}$;
\item define $A = M_{i_{k_1}} \ldots M_{i_{k_r}}$ and $B = M_{i_{j_1}} \ldots M_{i_{j_{p-r}}}$ (if $r=0$, $A=I$);
\item define $\epsilon_A = \epsilon_{i_{k_1}} \ldots \epsilon_{i_{k_r}}$ and $\epsilon_B = \epsilon_{i_{j_1}} \ldots \epsilon_{i_{j_{p-r}}} = \epsilon_A \epsilon_{i_1} \ldots \epsilon_{i_p}$.
\end{enumerate}
In particular, a choice of $A$ completely characterizes $B$. \newline
By varying $r$ from $0$ to $\left[\frac{p}{2}\right]$ and summing over all possible choices of $k_1 \ldots k_r$, every term in $\Tr D^p$ is generated.\newline
One can verify that every term in Eq.(\ref{eq:trd4}) ($p=4$) is of that type. For example:
\begin{align}
\Tr M_{i_1} [\epsilon_{i_1} &+ \epsilon_{i_2} \epsilon_{i_3} \epsilon_{i_4} *] \Tr ( M_{i_2} M_{i_3} M_{i_4}) = \notag \\
&\epsilon_{i_2} \epsilon_{i_3} \epsilon_{i_4} \Tr M_{i_1} \Tr (M_{i_2} M_{i_3} M_{i_4})^* + \epsilon_{i_1} \Tr M_{i_1} \Tr ( M_{i_2} M_{i_3} M_{i_4}) = \notag \\
&\epsilon_{i_2} \epsilon_{i_3} \epsilon_{i_4} \Tr M_{i_1} \Tr (M_{i_2} M_{i_3} M_{i_4})^* + \epsilon_{i_1} \Tr (M_{i_1})^* \Tr ( M_{i_2} M_{i_3} M_{i_4}) = \notag \\
&\Tr \big( \epsilon_{i_2} \epsilon_{i_3} \epsilon_{i_4}  M_{i_1} \otimes (M_{i_2} M_{i_3} M_{i_4})^* + \epsilon_{i_1}  M_{i_2} M_{i_3} M_{i_4} \otimes M_{i_1}^*  \big)
\end{align}
which is of the form of Eq.(\ref{eq:typical}) upon identifying $M_{i_1}$ with $A$ and $M_{i_2} M_{i_3} M_{i_4}$ with $B$ (in the second equality the reality of $\Tr M_{i_1}$ has been used). \newline
A way of expressing $\Tr B$ given $A$ is using a modified derivative operator $\D_i$ defined as:
\begin{equation}
\D_i \equiv \Tr \ \circ \ \frac{\partial}{\partial M_i} 
\end{equation}
which allows to write:
\begin{equation}
A = M_{i_{k_1}} \ldots M_{i_{k_r}} \implies \Tr B = \D_{i_{k_r}} \ldots \D_{i_{k_1}} \Tr (M_{i_1} \ldots M_{i_p}).
\end{equation}
Therefore Eq.(\ref{eq:typical}) becomes:
\begin{align}
\epsilon_A&[1+ \epsilon_{i_1} \ldots \epsilon_{i_p} *] (\Tr A)^* \Tr B = \notag \\
&\epsilon_{i_{k_1}} \ldots \epsilon_{i_{k_r}}[1 + \epsilon_{i_1} \ldots \epsilon_{i_p} * ] (\Tr M_{i_{k_1}} \ldots M_{i_{k_r}})^* \big( \D_{i_{k_r}} \ldots \D_{i_{k_1}} \Tr (M_{i_1} \ldots M_{i_p}) \big).
\end{align}
There are some special cases that make the expression simpler, namely:
\begin{enumerate}
\item $r=0$ gives a factor $\Tr I = n$;
\item $r=1, 2$ make $\Tr A$ real;
\item $p-r=1, 2$ (which can only occur for $p=2, 4$) make $\Tr B$ real. 
\end{enumerate}
Putting everything together, $\Tr D^p$ can be written as:
\begin{align}
\Tr D^p = \sum_{i_1 \ldots i_p} \Tr \ &\omega_{i_1} \ldots \omega_{i_p} \Bigg[ \sum_{r=0}^{\left[\frac{p}{2}\right]} \ \sum_{k_1 < \ldots < k_r = 1}^p  \epsilon_{i_{k_1}} \ldots \epsilon_{i_{k_r}} [1 + \epsilon_{i_1} \ldots \epsilon_{i_p} * ] \notag \\
&(\Tr M_{i_{k_1}} \ldots M_{i_{k_r}})^* \big( \D_{i_{k_r}} \ldots \D_{i_{k_1}} \Tr (M_{i_1} \ldots M_{i_p}) \big) \Bigg]. \notag \\
\end{align}
where:
\begin{align}
r=0 \quad &\longrightarrow \quad n[1+\epsilon_{i_1} \ldots \epsilon_{i_p}*]\Tr (M_{i_1} \ldots M_{i_p}) \\
r=1 \quad &\longrightarrow \quad \sum_{k_1 = 1}^p \epsilon_{i_{k_1}}  \Tr (M_{i_{k_1}}) [1+\epsilon_{i_1} \ldots \epsilon_{i_p}*] \D_{i_{k_1}} \Tr (M_{i_1} \ldots M_{i_p}) \\
r=2 \quad &\longrightarrow \quad \sum_{k_1<k_2 = 1}^p \epsilon_{i_{k_1}} \epsilon_{i_{k_2}}  \Tr (M_{i_{k_1}} M_{i_{k_2}}) [1+\epsilon_{i_1} \ldots \epsilon_{i_p}*] \D_{i_{k_2}} \D_{i_{k_1}} \Tr (M_{i_1} \ldots M_{i_p}).
\end{align}










\end{document}
